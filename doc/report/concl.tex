\section{Conclusions and Future Work}\label{conc}

\sys{} enables seamless mobility in Wi-Fi networks by providing a mechanism for lossless handoffs with zero latency. 
Hoolock accomplishes this by using two radios - scanning the available channels with one while 
maintaining an active connection over the other. Thus, when a hand-over is needed, it can do so immediately.
It has a superior performance compared to that of hard-handoff in terms of data loss during handoff. 

The performance of \sys{} has been evaluated using a simulation environment. 
In the real world, the wireless links are influenced by various factors, 
like multipath fading, Doppler shift and shadowing effects. 
In addition, the performance of the NOX controller, the client program and the bonding 
driver will be influenced by other system-wide factors. 
Our future work will consist of deployment of \sys{} on a real campus-wide network
and evaluation of its performance.

The Open-Flow infrastructure exhibits its power in our work. It has many advantages 
over the rigid, traditional networks. The major difference is its capability to 
manage the flow table entries in each switch, which provides administrators a
fine-grained control over the network traffic. The handoff 
application is just the tip of the iceberg - Open-Flow and its applications have 
a huge potential in defining the future Internet.
