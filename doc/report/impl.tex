\section{Implementation}\label{impl}

Figure~\ref{fig:sys} shows the design of the \sys{} system. It consists of three main parts - 
the bonding driver, the NOX module and the client program.

The bonding driver is a Linux kernel module that provides a single virtual interface for the upper 
layers to send packets to and receive packets from. 
It transparently manages the two physical wireless interfaces (called \emph{slaves})
to behave as required by the protocol described earlier.
It allows the client program to communicate to it using ioctl function calls, 
and provides two major functions:

\begin{itemize}
\item {\bf make()} - Start transmitting over the new slave while 
still listening on the old slave.
\item {\bf break()} - Stop listening on the old slave.
\end{itemize}

In addition to the above functions, the bonding driver also intelligently avoids 
any possible packet reordering due to rerouting of flows (especially, if the new
route has a lower latency than the old one). It does so by buffering packets received
on the new slave while in the make() mode and delivering them up when break() is called.

The NOX controller maintains a database of the currently active flows and their 
corresponding routes within the network.
It uses the Topology and Routing modules to route and switch flows.
Upon the receipt of a flow-in event from a client in transition, it calculates the corresponding route
to the destination, and uses it to calculate a corresponding \emph{reverse-route}. It then looks up
its database for all the routes with the client as the destination and matches them 
with the reverse-route to calculate the common \emph{root-switch}. It then, reroutes all the
flows destined to the client from root-switch onwards to follow the reverse-route, while updating the 
corresponding database entries.

The client program is the core module that coordinates between the bonding driver and the NOX
controller.
It communicates with bonding driver via the ioctl() function and interacts with NOX over a TCP channel
to execute the four-stage protocol described in Section~\ref{sec:design}.

